\chapter{引言}
\label{chap:introduction}

\section{研究背景与意义}
社交网络的兴起源于人类自身的社交需求。互联网技术的不断发展,使得线上的网络社交行为变的可能。线上的网络社交最早可追溯至电子邮件时期。1971年,人类的第一封电子邮件诞生,标志着人类彼此间的线上交流通道正式开启;20世纪80年代,电子公告牌系统(Bulletin Board System,BBS)上线并飞速发展,人们可以在BBS论坛中和其他用户一起讨论科学、文化和艺术等各方面的话题;1991年,万维网(World Wide Web, WWW)成立,进一步拉近了世界各地的距离。随着人类在互联网中的行为不断丰富,每个用户的个体形象也日趋丰满,真正意义上的在线社交网络开始慢慢浮现。2001年,Meetup.com网站\footnote{\url{https://www.meetup.com/}}上线,主要提供组织线下交友的功能;2002年,Friendster网站\footnote{\url{http://www.friendster.com/}}上线,开创了用户个人主页的先河;2003年,MySpace网站\footnote{\url{https://myspace.com/}}上线,直接刷新了社交网络的成长速度;2004年,Facebook网站\footnote{\url{http://www.facebook.com/}}在哈佛大学的寝室上线,并迅速席卷全球,目前已经成长为全世界最大的在线社交网络。此后,各类型的在线社交网络层出不穷,并且都迅速积累了庞大的用户量。2004年,图片社交平台Flickr\footnote{\url{https://www.flickr.com/}}上线;2005年,在线视频平台Youtube\footnote{\url{https://www.youtube.com/}}上线;2006年,短文本社交平台Twitter\footnote{\url{https://twitter.com/}}上线;2009年,基于地理位置的社交网络Foursquare\footnote{\url{https://foursquare.com/}}正式上线。国内的社交网络发展也紧跟世界的步伐,2005年,人人网\footnote{\url{http://www.renren.com}}(成立初始网站名为校内网)成立,在学生群体中掀起了一股风潮;2009年,新浪正式推出短文本社交平台新浪微博\footnote{\url{http://weibo.com}}。目前新浪微博已成为国内用户量最大的社交网络。

社交网络已经发展成为人们日常生活中不可或缺的一部分。Facebook发布的2017年第二季度业务报告\footnote{\url{https://investor.fb.com/investor-news/press-release-details/2017/Facebook-Reports-Second-Quarter-2017-Results/default.aspx}}中显示,网站的日活跃用户数超过13亿,月活跃用户数更是超过了20亿。新浪微博公布的《2016微博用户发展报告》\footnote{\url{http://data.weibo.com/report/reportDetail?id=346}}中显示,截止2016年9月30日,微博月活跃人数已达到2.97亿,日活跃用户达到1.32亿;在微博会员用户中,单月在线超过15天的会员用户占比达到84.8\%;微博月阅读量超过百亿的领域达到18个。这些数据都说明,社交网络在用户日常生活中的比重越来越高。因此,对社交网络展开一系列的研究,是非常有必要的。

随着互联网技术的不断发展,社交网络的功能也在不断演化。早期的社交网络主要提供的是网络社交的功能。社交网络平台为每个用户建立了主页,用户可以在自己的主页上更新自己的状态。用户之间通过关注关系或者好友关系形成网络。当用户更新自己的状态后,他所有的好友都可以接受到状态更新,进而了解他的近况。这样用户就可以在线维护自己的社交关系。随着时间的推移,用户发布的消息开始跳出个人状态的范畴,开始在社交网络上发布一些个人原创的内容,或是分享一些有价值的链接。借助于社交网络的``小世界"特性\citep{watts1998collective}(也称``无尺度性"\citep{barabasi1999emergence})\citep{java2007we},这些内容在社交网络平台上传播地非常迅速。创造优秀内容的用户也迅速积累了大量的粉丝,进而吸引了更多内容创造者的加入。社交网络的功能也从单纯的社交功能,开始向社交媒体转变\citep{hanna2011we,ellison2007social}。每一个社交网络平台的用户都可以成为内容的创造者,而社交网络结构则提供了类似``订阅"的功能:用户通过关注特定的用户来获得自己感兴趣的内容。社交网络平台在信息传播方面的优势,以及社交媒体中传播内容在形式上的多样性(文本、图片、音频、视频、地理位置等),更是吸引了一大批传统媒体入驻。传统的新闻媒体、广电媒体都开始进驻社交网络平台,在平台上发布自己的内容,如图\ref{fig:socialMedia}所示。社交网络的媒体功能,进一步加深了平台自身的用户粘性,吸引了更多的用户加入到社交网络中来,进而承载了更多在线内容的分享和交流。
\begin{figure}[!htb]
  \centering
  \begin{subfigure}[b]{0.4\textwidth}
    \includegraphics[width=\textwidth]{daily}
    \caption{人民日报}
  \end{subfigure}%
  \hspace{0.05\textwidth}
  \begin{subfigure}[b]{0.4\textwidth}
    \includegraphics[width=\textwidth]{cctv}
    \caption{央视网}
  \end{subfigure}
  \caption{传统媒体在微博平台上的公众号示例}
  \label{fig:socialMedia}
\end{figure}

社交网络自身规模和承载内容的不断增长,方便了用户信息获取的过程,但是也给带来了一些问题。社交网络中传播的内容呈现爆发式的增长,但是用户自身的时间和关注度是有限的,这带来了两方面的问题:用户角度,用户始终处于``信息过载"的状态。面对海量的信息,用户很难在其中发掘自己真正感兴趣的内容;内容角度,不同内容所获得的关注度也是不同的。因此,研究者也展开了一系列的研究,包括用户的采纳理论研究\citep{sledgianowski2008social,xu2008product,iyengar2011social,mansumitrchai2012factors}、传播理论研究\citep{bakshy2012role,gruhl2004information,guille2013information}以及流行度预测研究\citep{szabo2010predicting,pinto2013using}等。

在线内容的流行度是对内容受关注程度的一个度量。在不同的场景下,在线内容的具体形式不用,流行度可以有不同的量化方式。对于新闻类站点,在线内容主要以新闻为主,新闻的流行度可以用新闻页面的访问量和新闻的用户评论数来衡量;在视频类网站中,在线内容就是指网站上传的视频,它们的流行度可以用视频的播放量来衡量;在众多的社交平台上,在线内容的主要形式就是用户发布的文章(Facebook类平台)或者消息(Twitter类平台),内容的流行度可以使用丰富的用户交互信息来衡量。例如,在Twitter平台上,可以使用每一条消息所获得的转发数或者评论数来作为该消息的流行度指标;在Facebook上,可以使用每一篇文章获得的Like数或者评论数来作为该文章的流行度度量。

流行度预测研究的主要目标就是分析在线内容的流行度的变化趋势,建立模型来对内容后期的流行度进行预测,从而指导实际应用问题的解决方案的设计。流行度预测研究在实际生活中的很多领域都有着重要的意义。从用户的角度来看,流行度预测研究可以帮助用户在传播早期就发现潜在的热点内容\citep{tatar2014popularity,tatar2012ranking},起到信息过滤的作用;其次,对于内容提供商,流行度预测研究可以帮助站点定位可能的热点内容,辅助站点制定流量控制和资源分配策略\citep{chen2003popularity,famaey2013towards},从而保证热点内容的优先访问;在广告投放领域,流行度预测模型能够帮助广告主更好地指定广告的定价和投放策略;在市场营销领域,流行度预测研究可以帮助分析影响流行度传播的因素,进而辅助营销商家制定更好的营销方案。从科学研究的角度,在网络科学和社会学领域,流行度预测研究可以帮助人们更好地理解复杂网络系统中节点间的相互作用以及传播的动力学过程,推动网络科学和社会学领域的研究发展。

\section{研究现状}
目前社交网络中内容流行度的预测研究主要包含传播模型研究、用户影响力建模以及宏观层面的流行度预测方法等内容。本节将对这三部分内容进行简单的介绍,以方便读者更好地理解本论文的研究工作在流行度预测领域所处的位置以及本论文的主要贡献和创新之处。

传播模型方面,最早的传播建模方法来源于疾病传播模型,包括SIR(susceptible-infected-recovered)模型模型和SIS(susceptible-infected-susceptible)模型\citep{hill2010infectious}等。这类模型主要从宏观层面,来建模参与传播的用户数随时间的变化关系,但是对消息传播过程中的累积效应\citep{leskovec2006patterns}、爆发现象\citep{barabasi05}等特征不能很好地刻画。Kempe等人将用户间的个体交互行为类比为粒子系统中的粒子间的相互交互,粒子的激活与否对应用户是否参与了传播,并在此基础上提出了两种基础传播模型:独立级联模型(Independent Cascade,IC)和线性阈值模型(Linear Threshold,LT)\citep{kempe2003maximizing}。这两个模型从微观层面刻画了传播过程中的用户交互行为,在很多传播场景的问题中都有着重要的应用意义,包括影响力最大化问题\citep{wang2012scalable,cheng2013staticgreedy}和市场营销问题\citep{kim2014ct}等。

用户是社交网络的重要组成部分,也是影响社交网络中消息传播以及消息最终流行度的重要因素。早期对社交网络中用户影响力的刻画主要是从拓扑结构的角度展开的,将用户在网络中的出度、入度或是其他统计指标作为用户影响力的度量\citep{brown2011measuring,liang2012analyzing}。此外,社交网络中,用户的影响力也可以通过用户的活跃指数来表示\citep{li2013novel,cha2010measuring}。随着表达学习领域的兴起,也有工作开始将用户的影响力用向量来表示。Wang等人\citep{wang2015learning}将用户的影响力建模为影响力和易感度两个向量,用户间的影响概率由感染者的影响力向量和被感染者的易感度向量点积表示。这样的建模方式,大大减少了参数空间,同时考虑了同一用户与不同用户间的影响概率间的关联;Kurashima等人\citep{kurashima2014probabilistic}利用表达学习的方法,对用户的影响力进行了可视化展示;Bourigault等人\citep{bourigault2014learning}将传播过程类比为一个``热传导"过程,利用排序框架来学习传播用户的表达。

宏观层面的流行度预测方法主要包括三类:基于特征的有监督学习方法、基于随机过程的方法和基于表示学习的方法。本论文的第二章有关于这三类方法的详细阐述,这里只是对这三类方法进行简单的概括。基于特征的有监督学习方法通常将流行度预测问题形式化为回归或者分类问题,借助于现有的有监督学习框架和大量的历史数据,学习得到预测模型。这类模型的关键在于提取对流行度预测有指示意义的特征。常见的特征包括用户相关的特征、内容相关的特征、时序特征和传播结构特征等。基于随机过程的方法主要是将流行度的增长过程形式化为一个计数过程\citep{andersen1985counting},借助生存分析理论\citep{klein2005survival}的框架,来建模流行度的变化过程。这类方法的核心在于强度函数的设计。强度函数刻画了影响流行度变化的主要影响机制,常见的强度函数形式包括自增强泊松过程\citep{pemantle2007survey}和自激励Hawkes过程\citep{hawkes1974cluster}等。基于表示学习的方法则是利用深度学习模型,来学习传播过程中用户、消息或者是传播结构的表达,再对流行度的变化过程进行预测。除这三类模型外,还存在部分策略类的模型,包括融合基于特征的有监督学习方法和基于随机过程的方法、对消息和用户进行分组以及考虑消息间的相互竞争机制等。有关本部分的详细内容,请参阅本论文的第二章。

\section{面临的挑战}
尽管目前已经存在大量的流行度预测模型和方法,但是现有的模型和方法都存着各自的缺陷和不足。总结起来,目前在线社交网络中内容流行度预测研究仍然面临以下三个挑战:
\begin{itemize} 
\item \textbf{如何更好地利用海量的历史传播数据}:现有的流行度预测方法在对于历史传播数据的处理上都存在一定的缺陷。基于特征的有监督学习方法通常是将历史传播数据作为训练数据,训练得到一个预测模型,再将预测应用到新消息的预测上。但是历史传播数据中消息的传播情况差异非常大,导致学习得到的模型是一个全局层面折中后的结果;基于随机过程的方法则没有显式地利用数据,只是通过对历史数据的观测,来设计底层机制的形式,在模型的训练和预测过程中都没有利用到历史传播数据。历史传播数据中蕴含了丰富的消息传播模式和流行度变化信息,对这部分数据的充分挖掘有助于设计更精确的流行度预测方法。
\item \textbf{如何更好地刻画传播结构和用户影响}:在社交网络中,用户是消息转发的主体。不同的用户对消息传播过程的影响也不尽相同。在社交网络中,小度用户对消息的传播影响有限,但是影响力高的用户在参与到消息传播后,往往会带动他的粉丝也参与到消息的传播过程中,形成``二次传播"现象,最终使得消息的传播结构也区别于传统媒体中的单源传播结构。用户层面的影响异质性,是社交网络的本质特征之一。更好地刻画参与传播的用户对传播过程的影响,是实现精确的流行度预测目标的关键。
\item \textbf{如何建模时间不均匀性带来的影响}:无论是基于特征的有监督学习方法,还是基于随机过程的方法,时间尺度都是一个非常重要的因素。在特征类的方法中,时序特征本身就是一个非常重要的指示流行度变化的特征;在基于随机过程的方法中,流行度的到达时间更是模型的直接建模目标。在社交系统中,由于用户的使用习惯和作息习惯的影响,导致系统中的时间尺度是不均匀的:以微博系统为例,系统在白天新增的消息数目,要小于在晚上新增的消息数目,这是因为大部分用户白天都在上班的缘故;而凌晨到第二天早晨这一时间段内,系统的消息增长速度明显降低,这是由用户的作息时间决定的。这种时间尺度的变化,对于流行度预测有着重要的影响。但是现有的传播数据中,所记录的时间都是物理时间,在尺度上是均匀的。因此,建模系统的时间尺度的不均匀性,是流行度预测研究中的重要挑战。
\end{itemize}

\section{本文的工作}
\subsection{研究目标和内容}
本文的主要研究工作都是围绕社交系统中内容的流行度预测研究展开。本文将研究问题定位于``消息--用户--时间"结构的三维数据上,分别从消息、用户和时间三个维度,展开流行度预测研究。具体的研究内容包括:
\begin{enumerate}
\item 基于相似历史消息的流行度预测方法

历史消息的传播数据中包含了丰富的消息传播模式和流行度变化信息,对于流行度预测有着重要的指导意义,但已有的方法并没有对历史传播数据进行充分的挖掘和利用。本文从消息间的相似性出发,通过从历史消息中寻找与待预测消息的传播过程相似的消息,来预测待预测消息的流行度变化。本文研究了如何学习消息传播过程的表示,以及基于相似历史消息的流行度预测方法。
\item 中心用户引发的去中心化传播结构建模

不同于传统媒体中的中心型传播结构,社交网络中消息传播结构是去中心化的:部分影响力高的用户会引起消息传播的``二次爆发"。传统的预测方法将传播过程当做单源传播过程处理,而且没有精确地刻画用户影响力的异质性。本文对社交网络中消息的传播结构进行了分析,研究了高影响力用户在传播过程带来的激励作用,并对社交系统中消息的去中心化传播过程进行了建模。
\item 社交系统的时间尺度不均匀性建模

社交系统中时间尺度的不均匀性,对于消息的流行度预测有着重要的影响。传统的处理时间尺度不均匀性的方法主要是通过时间变换或周期函数的方式。本文研究了时间尺度在流行度增长过程中的影响,并将时间尺度与流行度增长过程中的时间效应函数融合,利用非参数化的方法,建模了时间尺度对流行度变化的影响。
\end{enumerate}
\subsection{研究成果}
本论文的主要研究成果包括:
\begin{enumerate}
\item 基于相似消息的流行度预测方法

借助于机器学习理论中近邻的思想,本文提出了一种基于相似消息的流行度预测方法。本文的方法先学习 得到消息传播过程的表示,再根据待预测消息和历史消息间的相似度,从历史数据选取与之传播过程相似的消息,并利用这些相似消息的数据,来对待预测消息的流行度变化进行预测。在度量消息间的相似性时,本文提出了一种基于转发时间间隔的方法。转发时间间隔可以反映消息中存在的不同的传播模式。借助于文本领域的主题模型,本文将每一条消息的传播过程当做文档,将消息的转发时间间隔作为单词,利用主题模型来学习得到消息的表示。根据待预测消息的表示和历史消息的表示,从历史消息中选出与之相似的前$K$条消息,再将这些消息的流行度的平均值作为预测结果。实验结果表明,相较于现有的方法,本文提出的方法的预测结果更精确、更稳定。
\item 面向多中心用户的流行度预测方法

本文提出了一种对基于自增强泊松过程的RPP模型\citep{shen2014modeling}进行叠加的方法来建模消息传播过程中的去中心化现象。本文首先对消息的去中心化传播过程进行了分析,发现消息的传播过程可以拆分为几个由中心用户引发的传播子过程的叠加,而每个中心用户引发的传播子过程都可以用现有的RPP模型较好地刻画。因此,本文提出了一种叠加RPP模型的方法来建模去中心化的传播过程,进而对消息的流行度作出预测。
\item 非参数化的时序释放函数建模

本文提出了一种非参数化的时序释放函数模型,用以建模社交系统中时间尺度不均匀性带来的影响。本文将流行度的增长过程形式化为用户影响力的释放过程。用户的影响力可以通过用户的粉丝数来体现,而释放过程可以用一个时序释放函数来刻画。时间尺度的不均匀性,为时序释放函数的设计和参数学习带来了挑战。因此,本文中提出了一种非参数化的方法,将社交系统中的物理时间划分为不同的时间段,并且为每一个时间段都学习得到一个对应的时序释放函数。数据集上的实验结果表明,我们的模型可以有效地提升流行度的预测效果。
\end{enumerate}
\subsection{论文的组织结构}
本论文的组织结构如下:

第一章介绍了本论文研究课题的研究背景、研究意义、研究现状以及本论文的研究目标、研究内容和主要研究成果。

第二章对流行度预测领域的相关工作进行了详细的综述。主要包括:(1)流行度的统计特征分析工作;(2)流行度预测模型和方法。涵盖了基于特征的有监督学习方法、基于随机过程的预测方法、基于表示学习的方法以及其他策略型的组合方法;(3)流行度预测问题的可预测性分析。

第三章介绍了基于时间间隔的消息表示学习方法,并在此基础上提出了基于相似消息的流行度预测模型。

第四章分析了社交网络中消息的传播过程,指出了消息传播过程中的去中心化现象,并提出了一种叠加RPP模型的方法来建模多中心用户引发的去中心化传播过程,进而对消息的流行度变化进行预测。

第五章提出了一种非参数化的方法,来建模时间尺度的不均匀性对流行度预测的影响。

第六章对本论文的整体研究工作进行了总结,指出了本论文的主要贡献和创新指出,并对本论文研究领域的未来发展方向进行了展望。
