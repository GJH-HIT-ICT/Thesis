\chapter{引言}
\label{chap:introduction}

\section{研究背景与意义}
社交网络的兴起源于人类自身的社交需求。互联网技术的不断发展,使得线上的网络社交行为变的可能。线上的网络社交最早可追溯至电子邮件时期。1971年,人类的第一封电子邮件诞生,标志着人类彼此间的线上交流通道正式开启;20世纪80年代,电子公告牌系统(Bulletin Board System,BBS)上线并飞速发展,人们可以在BBS论坛中和其他用户一起讨论科学、文化和艺术等各方面的话题;1991年,万维网(World Wide Web, WWW)成立,进一步拉近了世界各地的距离。随着人类在互联网中的行为不断丰富,每个用户的个体形象也日趋丰满,真正意义上的在线社交网络开始慢慢浮现。2001年,Meetup.com网站\footnote{\url{https://www.meetup.com/}}上线,主要提供组织线下交友的功能;2002年,Friendster网站\footnote{\url{http://www.friendster.com/}}上线,开创了用户个人主页的先河;2003年,MySpace网站\footnote{\url{https://myspace.com/}}上线,直接刷新了社交网络的成长速度;2004年,Facebook网站\footnote{\url{http://www.facebook.com/}}在哈佛大学的寝室上线,并迅速席卷全球,目前已经成长为全世界最大的在线社交网络。此后,各类型的在线社交网络层出不穷,并且都迅速积累了庞大的用户量。2004年,图片社交平台Flickr\footnote{\url{https://www.flickr.com/}}上线;2005年,在线视频平台Youtube\footnote{\url{https://www.youtube.com/}}上线;2006年,短文本社交平台Twitter\footnote{\url{https://twitter.com/}}上线;2009年,基于地理位置的社交网络Foursquare\footnote{\url{https://foursquare.com/}}正式上线。国内的社交网络发展也紧跟世界的步伐,2005年,人人网\footnote{\url{http://www.renren.com}}(成立初始网站名为校内网)成立,在学生群体中掀起了一股风潮;2009年,新浪正式推出短文本社交平台新浪微博\footnote{\url{http://weibo.com}}。目前新浪微博已成为国内用户量最大的社交网络。

社交网络已经发展成为人们日常生活中不可或缺的一部分。Facebook发布的2017年第二季度业务报告\footnote{\url{https://investor.fb.com/investor-news/press-release-details/2017/Facebook-Reports-Second-Quarter-2017-Results/default.aspx}}中显示,网站的日活跃用户数超过13亿,月活跃用户数更是超过了20亿。新浪微博公布的《2016微博用户发展报告》\footnote{\url{http://data.weibo.com/report/reportDetail?id=346}}中显示,截止2016年9月30日,微博月活跃人数已达到2.97亿,日活跃用户达到1.32亿;在微博会员用户中,单月在线超过15天的会员用户占比达到84.8\%;微博月阅读量超过百亿的领域达到18个。这些数据都说明,社交网络在用户日常生活中的比重越来越高。因此,对社交网络展开一系列的研究,是非常有必要的。

随着互联网技术的不断发展,社交网络的功能也在不断演化。早期的社交网络主要提供的是网络社交的功能。社交网络平台为每个用户建立了主页,用户可以在自己的主页上更新自己的状态。用户之间通过关注关系或者好友关系形成网络。当用户更新自己的状态后,他所有的好友都可以接受到状态更新,进而了解他的近况。这样用户就可以在线维护自己的社交关系。随着时间的推移,用户发布的消息开始跳出个人状态的范畴,开始在社交网络上发布一些个人原创的内容,或是分享一些有价值的链接。借助于社交网络的``小世界"特性\citep{watts1998collective}(也称``无尺度性"\citep{barabasi1999emergence})\citep{java2007we},这些内容在社交网络平台上传播地非常迅速。创造优秀内容的用户也迅速积累了大量的粉丝,进而吸引了更多内容创造者的加入。社交网络的功能也从单纯的社交功能,开始向社交媒体转变\citep{hanna2011we,ellison2007social}。每一个社交网络平台的用户都可以成为内容的创造者,而社交网络结构则提供了类似``订阅"的功能:用户通过关注特定的用户来获得自己感兴趣的内容。社交网络平台在信息传播方面的优势,以及社交媒体中传播内容在形式上的多样性(文本、图片、音频、视频、地理位置等),更是吸引了一大批传统媒体入驻。传统的新闻媒体、广电媒体都开始进驻社交网络平台,在平台上发布自己的内容,如图\ref{fig:socialMedia}所示。社交网络的媒体功能,进一步加深了平台自身的用户粘性,吸引了更多的用户加入到社交网络中来,进而承载了更多在线内容的分享和交流。
\begin{figure}[!htb]
  \centering
  \begin{subfigure}[b]{0.4\textwidth}
    \includegraphics[width=\textwidth]{daily}
    \caption{人民日报}
  \end{subfigure}%
  \hspace{0.05\textwidth}
  \begin{subfigure}[b]{0.4\textwidth}
    \includegraphics[width=\textwidth]{cctv}
    \caption{央视网}
  \end{subfigure}
  \caption{传统媒体在微博平台上的公众号示例}
  \label{fig:socialMedia}
\end{figure}

社交网络自身规模和承载内容的不断增长,方便了用户信息获取的过程,但是也给带来了一些问题。社交网络中传播的内容呈现爆发式的增长,但是用户自身的时间和关注度是有限的,这带来了两方面的问题:用户角度,用户始终处于``信息过载"的状态。面对海量的信息,用户很难在其中发掘自己真正感兴趣的内容;内容角度,不同内容所获得的关注度也是不同的。因此,研究者也展开了一系列的研究,包括用户的采纳理论研究\citep{sledgianowski2008social,xu2008product,iyengar2011social,mansumitrchai2012factors}、传播理论研究\citep{bakshy2012role,gruhl2004information,guille2013information}以及流行度预测研究\citep{szabo2010predicting,pinto2013using}等。

在线内容的流行度是对内容受关注程度的一个度量。在不同的场景下,在线内容的具体形式不用,流行度可以有不同的量化方式。对于新闻类站点,在线内容主要以新闻为主,新闻的流行度可以用新闻页面的访问量和新闻的用户评论数来衡量;在视频类网站中,在线内容就是指网站上传的视频,它们的流行度可以用视频的播放量来衡量;在众多的社交平台上,在线内容的主要形式就是用户发布的文章(Facebook类平台)或者消息(Twitter类平台),内容的流行度可以使用丰富的用户交互信息来衡量。例如,在Twitter平台上,可以使用每一条消息所获得的转发数或者评论数来作为该消息的流行度指标;在Facebook上,可以使用每一篇文章获得的Like数或者评论数来作为该文章的流行度度量。

流行度预测研究的主要目标就是分析在线内容的流行度的变化趋势,建立模型来对内容后期的流行度进行预测,从而指导实际应用问题的解决方案的设计。流行度预测研究在实际生活中的很多领域都有着重要的意义。从用户的角度来看,流行度预测研究可以帮助用户在传播早期就发现潜在的热点内容\citep{tatar2014popularity,tatar2012ranking},起到信息过滤的作用;其次,对于内容提供商,流行度预测研究可以帮助站点定位可能的热点内容,辅助站点制定流量控制和资源分配策略\citep{chen2003popularity,famaey2013towards},从而保证热点内容的优先访问;在广告投放领域,流行度预测模型能够帮助广告主更好地指定广告的定价和投放策略;在市场营销领域,流行度预测研究可以帮助分析影响流行度传播的因素,进而辅助营销商家制定更好的营销方案。从科学研究的角度,在网络科学和社会学领域,流行度预测研究可以帮助人们更好地理解复杂网络系统中节点间的相互作用以及传播的动力学过程,推动网络科学和社会学领域的研究发展。

\section{研究现状}

\section{面临的挑战}
\section{本文的工作}
\subsection{研究目标和内容}
\subsection{研究成果}
\subsection{论文的组织结构}
