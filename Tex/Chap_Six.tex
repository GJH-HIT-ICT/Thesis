\chapter{总结与展望}
\label{chap:six}

本章对论文工作进行总结,指出了论文的主要贡献和创新之处,并对下一步可能的研究工作进行了展望。
\section{本文主要贡献及创新之处}
随着互联网技术的不断发展,社交网络的功能也从传统的网络社交向社交媒体转变。越来越多的内容在社交网络中传播,也吸引了研究人员的广泛关注。流行度预测研究作为社交媒体领域的重要研究问题之一,在站点优化、广告投放以及市场营销等实际问题中有着重要的应用意义。本文将流行度预测问题定位于``消息--用户--时间"结构的三维数据上,从消息、用户、时间三个维度展开了研究。主要贡献及创新之处如下:
\begin{enumerate}
\item 基于相似消息的流行度预测方法

现有的方法没有对历史消息的传播数据进行充分的挖掘和利用。基于特征的有监督学习方法通常是将历史传播数据作为训练数据,训练得到一个预测模型,再将预测应用到新消息的预测上。但是历史传播数据中消息的传播情况差异非常大,导致学习得到的模型是一个全局层面折中后的结果;基于随机过程的方法则没有显式地利用数据,只是通过对历史数据的观测,来设计底层机制的形式,在模型的训练和预测过程中都没有利用到历史传播数据。借助于机器学习理论中近邻的思想,本文提出了一种基于相似消息的流行度预测方法。本文的方法先学习 得到消息传播过程的表示,再根据待预测消息和历史消息间的相似度,从历史数据选取与之传播过程相似的消息,并利用这些相似消息的数据,来对待预测消息的流行度变化进行预测。在度量消息间的相似性时,本文提出了一种基于转发时间间隔的方法。转发时间间隔可以反映消息中存在的不同的传播模式。借助于文本领域的主题模型,本文将每一条消息的传播过程当做文档,将消息的转发时间间隔作为单词,利用主题模型来学习得到消息的表示。根据待预测消息的表示和历史消息的表示,从历史消息中选出与之相似的前$K$条消息,再将这些消息的流行度的平均值作为预测结果。实验结果表明,相较于现有的方法,本文提出的方法的预测结果更精确、更稳定。
\item 面向多中心用户的流行度预测方法

本文首先对微博平台中消息的传播数据进行了分析,发现消息的传播过程呈现出去中心化的结构。对微博平台上的数据统计后发现,消息的源发用户引起的转发数占消息总转发数的比例有限,而部分中间用户引发的``二次传播"现象比较常见,因此,对消息的去中心化传播过程的建模是非常有必要的。通过对消息的去中心化传播过程的进一步分析,本文发现消息的传播过程可以拆分为几个由中心用户引发的传播子过程的叠加,而每个中心用户引发的传播子过程都可以用现有的RPP模型较好地刻画。因此,本文提出了一种叠加RPP模型的方法来建模去中心化的传播过程,进而对消息的流行度作出预测。与现有的方法相比,本文提出的方法放宽了RPP模型中的单源传播假设,而且能够刻画用户激励的差异性;与基于自激励Hawkes过程的方法相比,本文提出的方法在参数空间上有了大幅度的缩减,避免了基于自激励Hawkes过程的方法中存在的参数学习困难的问题。
\item 非参数化的时序释放函数建模

时间尺度的不均匀性是社交系统的固有特性,主要由系统中用户的使用习惯和作息规律导致的。现有的方法中对时间尺度不均匀性的建模主要是通过时间变换或周期函数的方式。本文提出了一种非参数化的时序释放函数模型,用以建模社交系统中时间尺度不均匀性带来的影响。本文将流行度的增长过程形式化为用户影响力的释放过程。用户的影响力可以通过用户的粉丝数来体现,而释放过程可以用一个时序释放函数来刻画。时间尺度的不均匀性,为时序释放函数的设计和参数学习带来了挑战。因此,本文中提出了一种非参数化的方法,将社交系统中的物理时间划分为不同的时间段,并且为每一个时间段都学习得到一个对应的时序释放函数。与现有的解读时间尺度不均匀性的方法相比,本文提出的方法更灵活,应用范围也更广。数据集上的实验结果表明,我们的模型可以有效地提升流行度的预测效果。
\end{enumerate}
\section{下一步的研究工作}
近几年来,流行度预测领域的工作层出不穷,但同时也面临着一些问题。首先需要解决的就是一个规范化的框架。Hofman等人\citep{hofman2017prediction}就指出,现有的流行度预测领域的工作中,数据、评测任务、评价指标以及数据处理过程,都不尽相同,这使得不同方法之间的比较难以进行,也不利于研究者分析不同算法各自的适用场景。因此,规范化的框架是目前需要解决的重要问题之一,也是本文下一步努力的目标。

其次,传播结构对流行度预测的影响目前尚不明确。现有的流行度预测方法中,对传播结构的处理通常是以特征提取的方式进行的。实际上,传播结构决定了消息的可见度和用户的影响力。因此,本文的下一步工作之一是研究传播结构对流行度变化的影响,进而实现更精确的流行度预测。

最后,由于社交系统的复杂性,流行度的增长过程是一个非常复杂的过程,这使得流行度预测问题的可预测性成为一个重要的议题。分析流行度预测问题的可预测性,可以帮助我们理解预测方法的表现和上限。因此,本文的后续工作中将对流行度预测问题的可预测性进行探讨。
